% !TEX TS-program = pdflatex
% !TEX encoding = UTF-8 Unicode

% This is a simple template for a LaTeX document using the "article" class.
% See "book", "report", "letter" for other types of document.

\documentclass[11pt]{article} % use larger type; default would be 10pt

\usepackage[utf8]{inputenc} % set input encoding (not needed with XeLaTeX)

%%% PAGE DIMENSIONS
\usepackage{geometry} % to change the page dimensions
\geometry{a4paper}

\usepackage{graphicx} % support the \includegraphics command and options

\usepackage[parfill]{parskip} % Begin paragraphs with an empty line

%%% PACKAGES
\usepackage{booktabs} % for much better looking tables
\usepackage{array} % for better arrays (eg matrices) in maths
\usepackage{paralist}
\usepackage{verbatim}
\usepackage{subfig}

%%% HEADERS & FOOTERS
\usepackage{fancyhdr} % This should be set AFTER setting up the page geometry
\pagestyle{fancy} % options: empty , plain , fancy
\renewcommand{\headrulewidth}{0pt} % customise the layout...
\lhead{}\chead{}\rhead{}
\lfoot{}\cfoot{\thepage}\rfoot{}

%%% SECTION TITLE APPEARANCE
\usepackage{sectsty}
\allsectionsfont{\sffamily\mdseries\upshape}
% (This matches ConTeXt defaults)

%%% ToC (table of contents) APPEARANCE
\usepackage[nottoc,notlof,notlot]{tocbibind} % Put the bibliography in the ToC
\usepackage[titles,subfigure]{tocloft} % Alter the style of Table of Contents
\renewcommand{\cftsecfont}{\rmfamily\mdseries\upshape}
\renewcommand{\cftsecpagefont}{\rmfamily\mdseries\upshape} % No bold!

\graphicspath{{figures/}}
\usepackage{color}

\usepackage{amsmath}
\usepackage{amsfonts}
\usepackage{hyperref}
% \usepackage{siunitx}
% \usepackage{authblk}

\usepackage{tikz}
\usepackage{standalone}

% \newtheorem{note}[theorem]{Note}

%%% END Article customizations

%%% The "real" document content comes below...

\title{Demo: Automate your Scientific Plots and Tables with Python and Make}
\author{Thomas P. Ogden}

%\date{} % Activate to display a given date or no date (if empty),
         % otherwise the current date is printed 

\begin{document}
\maketitle

The particular dataset and analysis here is not important, it's just to
have something to demo. We're looking at Transport for London's cycle hire usage
data aggregated in
\href{https://www.kaggle.com/hmavrodiev/london-bike-sharing-dataset/data}{this
Kaggle dataset}, for trips between 2015-01-04 and 2017-01-03.

\begin{figure}[h]
    \includegraphics[width=\textwidth]{figs/plot_daily_journeys.pdf}
    \caption{Average trips on weekdays by hour of day for dry and wet weather.}
    \label{fig:daily_journeys}
\end{figure}

\begin{table}[h]
  \centering
  % This table is auto-generated in src/tables and copied here.
  \input{figs/table_trips_per_hour.tex}
  \caption[]{Average trips on weekdays between 08:00-09:00 across seasons for
    dry and wet weather.}
  \label{tab:trips_per_hour}
\end{table}

\begin{figure}[h]
\includegraphics[width=\textwidth]{figs/plot_trips_vs_temp.pdf}
\caption{Average trips on weekdays between 08:00-09:00 versus the feels-like
  temperature.}
\label{fig:trips_vs_temp}
\end{figure}

% \bibliography{paper}
% \bibliographystyle{unsrt}

\end{document}

